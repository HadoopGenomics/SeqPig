
\section{Installation}

\Css{div.lstlisting{font-family: monospace;
    white-space: nowrap; margin-top:0.5em;
    margin-bottom:0.5em;
    color: blue;}}

\Css{ol li { padding: 0; margin: 0.5em; }}

\subsection{Dependencies}

\begin{enumerate}
	\item A Hadoop cluster (we have tested with Hadoop 0.20.2)
	\item Pig (at least version 0.10)
	\item Hadoop-BAM (\url{https://sourceforge.net/projects/hadoop-bam/})
	\item Seal (\url{http://biodoop-seal.sourceforge.net/})
\end{enumerate}

\subsection{Environment variables}
\label{sect:install_env}
\begin{enumerate}
\item Set Hadoop-related variables (e.g., {\tt HADOOP\_HOME}) for your
	installation
\item Set {\tt PIG\_HOME} to point to your Pig installation
\end{enumerate}

On a Cloudera Hadoop installation with Pig a suitable environment configuration would be:
\begin{lstlisting} 
export HADOOP_HOME=/usr/lib/hadoop
export PIG_HOME=/usr/lib/pig
\end{lstlisting}

\subsection{Instructions for building SeqPig}

\begin{enumerate}
\item Download hadoop-bam from \url{https://sourceforge.net/projects/hadoop-bam/}.

\item Download and build the latest Seal git master version from
 \url{http://biodoop-seal.sourceforge.net/}. Note that this requires setting
 {\tt HADOOP\_BAM} to the installation directory of hadoop-bam.

\item Inside the cloned SeqPig git repository create a
{\tt lib/} subdirectory and copy (or link) the jar files
from hadoop-bam and Seal to this new directory.  The files should be:
\begin{enumerate}
	\item \verb@${HADOOP_BAM}/*.jar@
	\item from the Seal directory, run \verb@ find build/ -name seal.jar@
\end{enumerate}
%
Note: the Picard and Sam jar files are contained in the hadoop-bam release
for convenience.

\item Run {\tt ant} to build {\tt SeqPig.jar}.
\end{enumerate}

Once you've built SeqPig, you can move the directory to a location of your
preference (if on a shared system, perhaps {\tt /usr/local/java/seqpig}, else
even your home directory could be fine).


Set the environment variable {\tt SEQPIG\_HOME} to point to the installation
directory of SeqPig; e.g.,
%
\begin{lstlisting} 
export SEQPIG_HOME=/usr/local/java/seqpig 
\end{lstlisting}
%

For your convenience, you can add the bin directory to your {\tt PATH}:
%
\begin{lstlisting} 
$ export PATH=${PATH}:${SEQPIG_HOME}/bin
\end{lstlisting}
%
This way, you'll be able to start a SeqPig-enabled Pig shell by running the {\tt
seqpig} command.

\subsubsection{Note}
\label{sect:piggybank_note}

Some of the example scripts in this manual (e.g.,
Section~\ref{sect:read_clipping}) require functions from \emph{PiggyBank},
which is a collection of publicly available User-Defined Functions (UDF's)
that are distributed with Pig but may need to be built separately, depending on
your Pig distribution.
For more details see
\url{https://cwiki.apache.org/confluence/display/PIG/PiggyBank}. Verify that
PiggyBank has been compiled by looking for the file {\tt piggybank.jar} under
{\tt \$PIG\_HOME}:
\begin{lstlisting} 
$ find $PIG_HOME -name piggybank.jar
\end{lstlisting}
If PiggyBank hasn't been compiled, go into {\tt
\$PIG\_HOME/contrib/piggybank/java} and run {\tt ant}.


\subsection{Usage}

\subsubsection{Pig grunt shell for interactive operations}
Assuming that all the environment variables have been set as described in the
previous sections, you can start the SeqPig-enabled ``grunt'' shell by running
%
\begin{lstlisting}
$ seqpig
\end{lstlisting}
%
\subsubsection{Starting scripts from the command line for non-interactive use}
Alternatively to using the interactive Pig grunt shell, users can write scripts
that are then submitted to Pig/Hadoop for automated execution. This type of
execution has the advantage of being able to handle parameters; for instance,
one can parametrize input and output files. See the {\tt /scripts} directory
inside the SeqPig distribution and Section \ref{sect:examples} for examples.
